\documentclass{aiaa-tc}% insert '[draft]' option to show overfull boxes

\usepackage{wrapfig}% embedding figures/tables in text (i.e., Galileo style)
\usepackage{threeparttable}% tables with footnotes
\usepackage{dcolumn}% decimal-aligned tabular math columns
\newcolumntype{d}{D{.}{.}{-1}}
\usepackage{nomencl}% automatic nomenclature generation via makeindex
\makeglossary
\usepackage{subfigure}% subcaptions for subfigures
\usepackage{subfigmat}% matrices of similar subfigures, aka small mulitples
\usepackage{fancyvrb}% extended verbatim environments
\fvset{fontsize=\footnotesize,xleftmargin=2em}
\usepackage{lettrine}% dropped capital at beginning of paragraph
%\usepackage[dvips]{dropping}% alternative dropped capital package

\usepackage{booktabs}
\usepackage{bm, amssymb, amsmath, array, pdfpages}
\usepackage{amsmath,amssymb,color,graphicx,pdfsync,overpic,color,epstopdf,rotating,dashrule,float,bm}
\usepackage{float}
\usepackage{hyperref}

\usepackage{setspace}
%\doublespacing
\usepackage{comment}
\usepackage{changepage}

\usepackage{titlesec}

\graphicspath{{Figures/}}

\title{Quantifying the Effects of Sensor Noise on Dynamic Mode Decomposition Models}

\author{%
  Scott Dawson\thanks{Department of Mechanical and Aerospace
    Engineering; Student Member, AIAA.}
  \ ,
  Anthony M. DeGennaro\thanks{Department of Mechanical and Aerospace
    Engineering; Student Member, AIAA.}
  \ ,
  Clarence W. Rowley\thanks{Department of Mechanical and Aerospace Engineering;
    Associate Fellow, AIAA.}\\
  {\normalsize\itshape
   Princeton University, Princeton, NJ, 08540, USA}\\
   %{\normalsize \copyright~2014 by the authors. Do not distribute without permission.}
 }

 \AIAApapernumber{200?-????}
 \AIAAconference{Conference Name, Date, and Location}
 \AIAAcopyright{\AIAAcopyrightD{200?}}

 % define some commands to maintain consistency
 \newcommand{\pkg}[1]{\texttt{#1}}
 \newcommand{\cls}[1]{\textsf{#1}}
 \newcommand{\file}[1]{\texttt{#1}}
\newcommand{\todo}[1]{\textcolor{red}{[TODO: #1]}}
\newcommand{\cwrremark}[1]{\textcolor{blue}{[CWR: #1]}}
\newcommand{\R}{\mathbb{R}}
\newcommand{\CLmax}{{C_L}_\text{max}}
\newcommand{\LDmax}{L/D_\text{max}}
\newcommand{\alphamax}{\alpha_\text{max}}
\newcommand*{\mat}[1]{{\bf{#1}}}
\def\ip<#1,#2>{\left\langle #1,#2\right\rangle}

\begin{document}

\maketitle

\begin{abstract}
  The purpose of this study is to explore and quantify the statistical
  effect that sensor noise has on system identification. Specifically,
  we are interested in linear models empirically identified using
  Dynamic Mode Decomposition (DMD), and the effects that noise of
  varying bias and strength can have on the eigenvalues of that linear
  system. We propose a framework for investigating this, drawing our
  tools from the uncertainty quantification (UQ) community.

\end{abstract}

\section{Introduction}

[Some shit about DMD, how it is affected by noise, why that is
  important to study, etc... big picture stuff.]

\section{System Identification using Dynamic Mode Decomposition (DMD)}

[Scott can write some standard shit here, just to introduce the
  tools/concepts].

\section{Uncertainty Quantification using Polynomial Chaos Expansions (PCE)}

We would like to vary the bias and strength of noise on different
combinations of states and quantify how this affects the eigenvalues
of the DMD matrix. Additionally, we would like to obtain a surrogate
model that explicitly describes this input-output behavior, which can
be inexpensively sampled and analyzed for statistical correlations and
trends. As a result of these requirements, we choose to use Polynomial
Chaos Expansions (PCE) as our framework for performing UQ.

We give a brief overview of this approach below; further details can
be found in introductory references on PCE
methods\cite{ghanem_book,xiu_book,lemaitre}.

Let~$Z=(Z_1,\ldots,Z_d)$ be a vector of random variables that
parameterize the uncertain quantities in the sensor noise. We are
interested in the corresponding uncertainty of an aerodynamic
quantity, represented by $y(Z)$.  In our setting, $Z$ are the mean and
variance of the noise present on each state, and $y(Z)$ will be the
mean and covariance of the DMD eigenvalues.

The goal of the method is to represent $y(Z)$ in terms of some basis
functions~$\Phi_i$. Assuming (for ease of exposition) that $y(Z)$ is
scalar-valued, we write:
\begin{equation}
  \label{eq:1}
  y(Z) = \sum_{|i|=0}^N y_i \Phi_i(Z).
\end{equation}
Here, $i=(i_1,\ldots,i_d)$ is a multi-index, and $|i|=\sum_{j=1}^d i_j$.  We define an inner
product on the space of functions of the random variables by
\begin{equation}
  \label{eq:2}
  \ip<f,g> = \int_\Gamma f(Z) g(Z) \rho(Z)\,dZ,
\end{equation}
where $\rho(Z)$ denotes the probability density function of $Z$, and
has support $\Gamma$.  A fundamental insight in PCE methods is to
employ basis functions that are orthonormal with respect to this
inner product, so that
\begin{equation}
  \label{eq:3}
  \ip<\Phi_i,\Phi_j> = \delta_{ij},
\end{equation}
where $\delta_{ij}=1$ if $i=j$, and $0$ if $i\ne j$. In particular,
a multivariate basis polynomial $\Phi_i$ may be written as
\begin{equation}
\Phi_i(Z) = \prod_{k=1}^d \phi_{i_k}(Z_k),
\end{equation}
where $\phi_n$ is a (univariate) polynomial of degree~$n$. The $\{ \phi_n\}$ will be a basis of
orthogonal polynomials chosen so that the orthogonality
condition~\eqref{eq:3} is satisfied. In this paper, we work
exclusively with uniformly distributed random variables, and so our
basis polynomials are the multivariate Legendre polynomials.

The coefficients $y_i$ in the expansion~\eqref{eq:1} may be determined
by taking an inner product with $\Phi_j$: because the $\Phi_j$ are
orthonormal, we have
\begin{equation}
  \label{eq:4}
  y_j = \ip<y,\Phi_j>.
\end{equation}
Note that one could also take $y(Z)$ to be a vector of several different
aerodynamic quantities of interest: in this case, the coefficients~$y_i$ in the
expansion~\eqref{eq:1} are vectors, and each component of $y_i$ is determined by
an equation such as~\eqref{eq:4}, for the corresponding component of~$y$.

The important issue now is how we choose to approximate the projection
integrals in~\eqref{eq:4}. A possible choice is to use Gauss
quadrature, in which the function $y(Z)$ is evaluated on a grid
consisting of the tensor product of $n$ separate 1-D quadrature point
sets in parameter space. However, this method suffers from the
curse of dimensionality, since the number of required samples grows
exponentially with the dimension~$n$.

A commonly used alternative is to use sparse grid methods\cite{Smolyak}, in
which the number of grid points used is lessened by using only a subset of the
full tensor product.
%
Another advantage is that anisotropic
adaptive $p$-refinement of the mesh is possible, since nested 1-D
nodes are used. In an adaptive setting, global sensitivities are
calculated using total Sobol indices, which are defined for each
parameter as:
\begin{equation}
T_i = \frac{\mathbb{E} [ Var(y|Z_{-i}) ]}{Var(y)},\qquad i = 1,\ldots,d,
\label{eq:sobol}
\end{equation}
Here,
$Var(y|Z_{-i})$ denotes the variance of $y(Z)$ given all parameters
except $Z_i$. This is a measure of ``how much'' parameter $Z_i$
contributes to the total variance of $y(Z)$ on average. Parameters
that have higher Sobol indices contribute more to the variance of the
response and hence require more refinement than parameters with lower
Sobol indices.

Further details on the sparse grid approach can be found in standard
references\cite{lemaitre,Gerstner:SparseGrids}. In our study, we compute the
sparse grids using DAKOTA, an
open-source code for optimization and UQ developed by Sandia National
Laboratory~\cite{Dakota}.

\section{Quantifying the Effect of Sensor Noise on DMD Eigenvalues}

Basically, just make the point that our inputs in the UQ framework are
the bias and scaling of a multivariate noise distribution (eg.,
Gaussian) defined over all of our states, and the outputs are some
measures of the empirical eigenvalue statistics (eg., mean and
covariance in the complex plane of the eigenvalue locations).

I can even hash together a simple example if I get my shit together.



\bibliographystyle{aiaa}	% (uses file "plain.bst")
\bibliography{DMDUQreferences}		% expects file "gPC_Bib.bib"

\end{document}

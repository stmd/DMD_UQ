\documentclass[12pt]{article}
\usepackage{amsmath,amscd,amssymb,graphicx,latexsym,epstopdf,overpic,multicol,natbib,rotating,setspace}
\usepackage[T1]{fontenc}
\usepackage{color}
\newcommand{\td}[1]{\tilde{#1}}
\newcommand{\pd}{\partial}
\newcommand{\bpv}[2]{\begin{pmatrix}#1\\ #2\end{pmatrix}}
\newcommand{\pdfrac}[2]{\frac{\partial #1}{\partial #2}}
\newcommand{\pdfractwo}[2]{\frac{\partial^2 #1}{\partial #2^2}}
\newcommand{\Htwo}{\mathcal{H}_2}
\newcommand{\Hinf}{\mathcal{H}_\infty}
\newcommand\Real{\mbox{Re}} % cf plain TeX's \Re and Reynolds number
\newcommand\Imag{\mbox{Im}} % cf plain TeX's \Im
\newcommand\Rey{\mbox{\textit{Re}}}  % Reynolds number
\newcommand\Ma{\mbox{\textit{Ma}}}  % Reynolds number
\newcommand\Str{\mbox{\textit{St}}}
\newcommand\Pran{\mbox{\textit{Pr}}} % Prandtl number, cf TeX's \Pr product
\newcommand\Pen{\mbox{\textit{Pe}}}  % Peclet number
\newcommand\Ai{\mbox{Ai}}            % Airy function
\newcommand\Bi{\mbox{Bi}}            % Airy function
\newcommand\ssC{\mathsf{C}}    % for sans serif C
\newcommand\sfsP{\mathsfi{P}}  % for sans serif sloping P
\newcommand\slsQ{\mathsfbi{Q}} % for sans serif bold-sloping Q
\newcommand\hatp{\skew3\hat{p}}      % p with hat
\newcommand\hatR{\skew3\hat{R}}      % R with hat
\newcommand\hatRR{\skew3\hat{\hatR}} % R with 2 hats
\newcommand\doubletildesigma{\skew2\tilde{\skew2\tilde{\Sigma}}}
\newsavebox{\astrutbox}
\sbox{\astrutbox}{\rule[-5pt]{0pt}{20pt}}
\newcommand{\astrut}{\usebox{\astrutbox}}

\newcommand\GaPQ{\ensuremath{G_a(P,Q)}}
\newcommand\GsPQ{\ensuremath{G_s(P,Q)}}
\newcommand\p{\ensuremath{\partial}}
\newcommand\tti{\ensuremath{\rightarrow\infty}}
\newcommand\kgd{\ensuremath{k\gamma d}}
\newcommand\shalf{\ensuremath{{\scriptstyle\frac{1}{2}}}}
\newcommand\sh{\ensuremath{^{\shalf}}}
\newcommand\smh{\ensuremath{^{-\shalf}}}
\newcommand\squart{\ensuremath{{\textstyle\frac{1}{4}}}}
\newcommand\thalf{\ensuremath{{\textstyle\frac{1}{2}}}}
\newcommand\Gat{\ensuremath{\widetilde{G_a}}}
\newcommand\ttz{\ensuremath{\rightarrow 0}}
\newcommand\ndq{\ensuremath{\frac{\mbox{$\partial$}}{\mbox{$\partial$} n_q}}}
\newcommand\sumjm{\ensuremath{\sum_{j=1}^{M}}}
\newcommand\pvi{\ensuremath{\int_0^{\infty}%
  \mskip \ifCUPmtlplainloaded -30mu\else -33mu\fi -\quad}}

\newcommand\etal{\mbox{\textit{et al.}}}
\newcommand\etc{etc.\ }
\newcommand\eg{e.g.\ }
\title{Uncertainty quantification of the dynamic mode decomposition}

\author{%
  Anthony M. DeGennaro, 
  Scott T. M. Dawson, \\ and
  Clarence W. Rowley\\
  \itshape
   Princeton University\\
 }

 % define some commands to maintain consistency
 \newcommand{\pkg}[1]{\texttt{#1}}
 \newcommand{\cls}[1]{\textsf{#1}}
 \newcommand{\file}[1]{\texttt{#1}}
\newcommand{\todo}[1]{\textcolor{red}{[TODO: #1]}}
\newcommand{\cwrremark}[1]{\textcolor{blue}{[CWR: #1]}}
\newcommand{\R}{\mathbb{R}}
\newcommand{\CLmax}{{C_L}_\text{max}}
\newcommand{\LDmax}{L/D_\text{max}}
\newcommand{\alphamax}{\alpha_\text{max}}
\newcommand*{\mat}[1]{{\bf{#1}}}
\def\ip<#1,#2>{\left\langle #1,#2\right\rangle}

\begin{document}

\maketitle

%\begin{abstract}
 This work explores and quantifies the statistical effect that
 parameterized uncertainty has on the dynamic mode decomposition
 (DMD).  For the data under consideration, such uncertain parameters
 could include Reynolds number, geometry, or random {\color{green}
   sensor/signal} noise in the system.  The aims of this study are
 twofold: firstly, to quantify the robustness of the algorithm in
 terms of pertinent identified quantities (such as DMD modes and
 eigenvalues), and secondly, to present a method for analyzing the
 underlying dynamic systems from data in an efficient
 manner. {\color{green} We use polynomial chaos expansions to
   represent the relevant DMD quantities of interest. This approach}
 can be computationally more efficient than sample-based methods
 (e.g., Monte Carlo) when the dimensionality of the parameter space is
 moderate. Additionally, polynomial chaos methods yield a polynomial
 function for a surrogate model, which makes trivial tasks such as
 evaluating the surrogate, investigating statistical trends, and
 evaluating sensitivities.  We demonstrate our methodology on a number
 of well-studied example systems, including numerical simulations of
 flow past a circular cylinder. {\color{green} The main contribution
   of this work is to demonstrate a method for quantifying uncertainty
   in DMD models, given {\it a priori} knowledge of uncertainty in
   governing parameters or noise in the data.}

%\end{abstract}

\end{document}

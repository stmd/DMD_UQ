\documentclass[12pt]{article}
\usepackage{amsmath,amscd,amssymb,graphicx,latexsym,epstopdf,overpic,multicol,natbib,rotating,setspace}
\usepackage[T1]{fontenc}
\usepackage{color}
\newcommand{\td}[1]{\tilde{#1}}
\newcommand{\pd}{\partial}
\newcommand{\bpv}[2]{\begin{pmatrix}#1\\ #2\end{pmatrix}}
\newcommand{\pdfrac}[2]{\frac{\partial #1}{\partial #2}}
\newcommand{\pdfractwo}[2]{\frac{\partial^2 #1}{\partial #2^2}}
\newcommand{\Htwo}{\mathcal{H}_2}
\newcommand{\Hinf}{\mathcal{H}_\infty}
\newcommand\Real{\mbox{Re}} % cf plain TeX's \Re and Reynolds number
\newcommand\Imag{\mbox{Im}} % cf plain TeX's \Im
\newcommand\Rey{\mbox{\textit{Re}}}  % Reynolds number
\newcommand\Ma{\mbox{\textit{Ma}}}  % Reynolds number
\newcommand\Str{\mbox{\textit{St}}}
\newcommand\Pran{\mbox{\textit{Pr}}} % Prandtl number, cf TeX's \Pr product
\newcommand\Pen{\mbox{\textit{Pe}}}  % Peclet number
\newcommand\Ai{\mbox{Ai}}            % Airy function
\newcommand\Bi{\mbox{Bi}}            % Airy function
\newcommand\ssC{\mathsf{C}}    % for sans serif C
\newcommand\sfsP{\mathsfi{P}}  % for sans serif sloping P
\newcommand\slsQ{\mathsfbi{Q}} % for sans serif bold-sloping Q
\newcommand\hatp{\skew3\hat{p}}      % p with hat
\newcommand\hatR{\skew3\hat{R}}      % R with hat
\newcommand\hatRR{\skew3\hat{\hatR}} % R with 2 hats
\newcommand\doubletildesigma{\skew2\tilde{\skew2\tilde{\Sigma}}}
\newsavebox{\astrutbox}
\sbox{\astrutbox}{\rule[-5pt]{0pt}{20pt}}
\newcommand{\astrut}{\usebox{\astrutbox}}

\newcommand\GaPQ{\ensuremath{G_a(P,Q)}}
\newcommand\GsPQ{\ensuremath{G_s(P,Q)}}
\newcommand\p{\ensuremath{\partial}}
\newcommand\tti{\ensuremath{\rightarrow\infty}}
\newcommand\kgd{\ensuremath{k\gamma d}}
\newcommand\shalf{\ensuremath{{\scriptstyle\frac{1}{2}}}}
\newcommand\sh{\ensuremath{^{\shalf}}}
\newcommand\smh{\ensuremath{^{-\shalf}}}
\newcommand\squart{\ensuremath{{\textstyle\frac{1}{4}}}}
\newcommand\thalf{\ensuremath{{\textstyle\frac{1}{2}}}}
\newcommand\Gat{\ensuremath{\widetilde{G_a}}}
\newcommand\ttz{\ensuremath{\rightarrow 0}}
\newcommand\ndq{\ensuremath{\frac{\mbox{$\partial$}}{\mbox{$\partial$} n_q}}}
\newcommand\sumjm{\ensuremath{\sum_{j=1}^{M}}}
\newcommand\pvi{\ensuremath{\int_0^{\infty}%
  \mskip \ifCUPmtlplainloaded -30mu\else -33mu\fi -\quad}}

\newcommand\etal{\mbox{\textit{et al.}}}
\newcommand\etc{etc.\ }
\newcommand\eg{e.g.\ }
\title{Quantifying the Effects of Sensor Noise on Dynamic Mode Decomposition Models}

\author{%
  Anthony M. DeGennaro, 
  Scott T. M. Dawson, \\ and
  Clarence W. Rowley\\
  \itshape
   Princeton University\\
 }

 % define some commands to maintain consistency
 \newcommand{\pkg}[1]{\texttt{#1}}
 \newcommand{\cls}[1]{\textsf{#1}}
 \newcommand{\file}[1]{\texttt{#1}}
\newcommand{\todo}[1]{\textcolor{red}{[TODO: #1]}}
\newcommand{\cwrremark}[1]{\textcolor{blue}{[CWR: #1]}}
\newcommand{\R}{\mathbb{R}}
\newcommand{\CLmax}{{C_L}_\text{max}}
\newcommand{\LDmax}{L/D_\text{max}}
\newcommand{\alphamax}{\alpha_\text{max}}
\newcommand*{\mat}[1]{{\bf{#1}}}
\def\ip<#1,#2>{\left\langle #1,#2\right\rangle}

\begin{document}

\maketitle

\begin{abstract}
 This study is to explores and quantifies the statistical
  effect that parameterized uncertainty has on system
  identification. Specifically, we focus on the dynamic mode decomposition (DMD), and consider
  the effects that statistical variation in governing parameters
  can have on identified quantities such as DMD modes and eigenvalues. 
  %Much is known
  %about how to identify linear models from data collected from a
  %particular dynamical system using DMD; what we wish to investigate
  %is how these models vary given some prescribed statistical
  %uncertainty in governing parameters, such . 
  We propose a framework for
  investigating this, which is borrowed from the uncertainty
  quantification (UQ) community, called polynomial chaos expansions
  (PCE). This method is advantageous for two reasons. First, PCE
  methods can be computationally more efficient than sample-based
  methods (e.g., Monte Carlo) at when the dimensionality of the
  parameter space is moderate. Second, PCE methods yield a polynomial
  function for a surrogate model, which makes trivial tasks such as
  evaluating the surrogate, investigating statistical trends, and
  evaluating sensitivities.
  We demonstrate our method on a number of well-studied example systems, including data from the 
  Stuart-Landau equation, and DNS data for 2-D flow past a circular
  cylinder. 
 % These examples are chosen because they have been
  %extensively studied (thus making it easy to assess the efficacy
  %of our method) have dynamical behavior which varies as
  %certain governing parameters are changed.

\end{abstract}

\end{document}

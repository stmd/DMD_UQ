\documentclass{article}% insert '[draft]' option to show overfull boxes

\usepackage{wrapfig}% embedding figures/tables in text (i.e., Galileo style)
\usepackage{threeparttable}% tables with footnotes
\usepackage{dcolumn}% decimal-aligned tabular math columns
\newcolumntype{d}{D{.}{.}{-1}}
\usepackage{nomencl}% automatic nomenclature generation via makeindex
\makeglossary
\usepackage{subfigure}% subcaptions for subfigures
\usepackage{subfigmat}% matrices of similar subfigures, aka small mulitples
\usepackage{fancyvrb}% extended verbatim environments
\fvset{fontsize=\footnotesize,xleftmargin=2em}
\usepackage{lettrine}% dropped capital at beginning of paragraph
%\usepackage[dvips]{dropping}% alternative dropped capital package

\usepackage{booktabs}
\usepackage{bm, amssymb, amsmath, array, pdfpages}
\usepackage{amsmath,amssymb,color,graphicx,pdfsync,overpic,color,epstopdf,rotating,dashrule,float,bm}
\usepackage{float}
\usepackage{hyperref}

\usepackage{setspace}
%\doublespacing
\usepackage{comment}
\usepackage{changepage}

\usepackage{titlesec}

\graphicspath{{Figures/}}

\title{Quantifying the Effects of Sensor Noise on Dynamic Mode Decomposition Models}

\author{%
  Anthony M. DeGennaro\thanks{Department of Mechanical and Aerospace
    Engineering; Student Member, AIAA.}
  \ ,
  Scott Dawson\thanks{Department of Mechanical and Aerospace
    Engineering; Student Member, AIAA.}
  \ ,
  Clarence W. Rowley\thanks{Department of Mechanical and Aerospace Engineering;
    Associate Fellow, AIAA.}\\
  {\normalsize\itshape
   Princeton University, Princeton, NJ, 08540, USA}\\
   %{\normalsize \copyright~2014 by the authors. Do not distribute without permission.}
 }

 % define some commands to maintain consistency
 \newcommand{\pkg}[1]{\texttt{#1}}
 \newcommand{\cls}[1]{\textsf{#1}}
 \newcommand{\file}[1]{\texttt{#1}}
\newcommand{\todo}[1]{\textcolor{red}{[TODO: #1]}}
\newcommand{\cwrremark}[1]{\textcolor{blue}{[CWR: #1]}}
\newcommand{\R}{\mathbb{R}}
\newcommand{\CLmax}{{C_L}_\text{max}}
\newcommand{\LDmax}{L/D_\text{max}}
\newcommand{\alphamax}{\alpha_\text{max}}
\newcommand*{\mat}[1]{{\bf{#1}}}
\def\ip<#1,#2>{\left\langle #1,#2\right\rangle}

\begin{document}

\maketitle

\begin{abstract}
  The purpose of this study is to explore and quantify the statistical
  effect that parameterized uncertainty has on system
  identification. Specifically, we are interested in linear models
  empirically identified using Dynamic Mode Decomposition (DMD), and
  the effects that statistical variation in the governing parameters
  can have on the eigenvalues of that linear system. Much is known
  about how to identify linear models from data collected from a
  particular dynamical system using DMD; what we wish to investigate
  is how these models vary given some prescribed statistical
  uncertainty in governing parameters. We propose a framework for
  investigating this, which is borrowed from the uncertainty
  quantification (UQ) community, called Polynomial Chaos Expansions
  (PCE). This method is advantageous for two reasons. First, PCE
  methods can be computationally more efficient than sample-based
  methods (eg. Monte Carlo) at when the dimensionality of the
  parameter space is moderate. Second, PCE methods yield a polynomial
  function for a surrogate model, which makes trivial tasks such as
  evaluating the surrogate, investigating statistical trends,
  evaluating sensitivities, etc.

  We will demonstrate our method on some example systems, such as the
  Stuart-Landau dynamical system, as well as 2-D flow past a
  cylinder. These examples are chosen because they have been
  extensively studied, which will make it easy to assess the efficacy
  of our method. Both examples have dynamical behavior which varies as
  certain governing parameters are changed.

\end{abstract}

\end{document}
